%-------------------------------------------------------------------------------
%	SECTION TITLE
%-------------------------------------------------------------------------------
\cvsection{Course Projects}


%-------------------------------------------------------------------------------
%	CONTENT
%-------------------------------------------------------------------------------
\begin{cventries}

%---------------------------------------------------------
  \cventry
    {JAVA | Data Structure} % Job title
    {Image Compression System - operating Data Structures} % Organization
    {Prof. Mausam } % Location
    {September’17} % Date(s)
    {
      \begin{cvitems} % Description(s) of tasks/responsibilities
        \item {To encode a given image in a compressed format and perform operations like Inversion and Changing+updating pixel values and perform basic binary opertions like AND, OR and XOR.}
        \item {Idea of the compressed representation of any image
was to use the redundancy in the pixel value information
among neighbouring pixels to
reduce the amount of information
that needs to be stored.}
      \end{cvitems}
    }

%---------------------------------------------------------
  \cventry
    {Digital Logic | XilinxISE(VHDL)  } % Job title
    {Universal Asynchronous Receiver-Transmitter(UART)} % Organization
    {Prof. Anshul Kumar} % Location
    {Oct Nov 2017} % Date(s)
    {
      \begin{cvitems} % Description(s) of tasks/responsibilities
        \item {Establishing a UART protocol between PC and FPGA (Basys3 Board) to perform serial communication by transmitting data at some give frequency.}
        \item { Configuring the data-format and transmission speeds in a hardware device for asynchronous serial communication in a UART. }
        \item {Performing communication through data-stream of the characters in
their ASCII format.}
      \end{cvitems}
    }

%---------------------------------------------------------
  \cventry
    {JAVA | Graph Theory + Data-Structure} % Job title
    {Shortest Path Problem - Dijkstra's Algorithm} % Organization
    {Prof. Mausam
} % Location
    { November, 2017} % Date(s)
    {
      \begin{cvitems} % Description(s) of tasks/responsibilities
        \item {Implemented a Dijkstra’s algorithm for finding the shortest path between two configurations
of an Puzzle-tile game.}
        \item {Takes in some intial
and final board configurations
with varying path costs and outputs the minimum path and its cost.}
      \end{cvitems}
    }

%---------------------------------------------------------
  \cventry
    {Xilinx ISE(VHDL) + Digital Logic} % Job title
    {Designed an Elevator System} % Organization
    {Prof. Anshul Kumar} % Location
    {October, 2017} % Date(s)
    {
      \begin{cvitems} % Description(s) of tasks/responsibilities
        \item {Implemented on the FPGA (Basys3) board using VHDL comprising of a model of two lifts
spanning over 4 floors.}
        \item {Handled multiple requests from inside the lifts as well as from the respective floor using Fixed State Machine(FSM) while the current state of both the lifts
was displayed on a Seven Segment Display(SSD), which itself was designed in VHDL schematic.
}
      \end{cvitems}
    }

%---------------------------------------------------------
  \cventry
    {JAVA | Hashing + Data Indexing in B‑Trees } % Job title
    {Data-Structure and Algorithm Assignments} % Organization
    {Prof. Mausam} % Location
    {JUly'17 - Novemeber'17} % Date(s)
    {
      \begin{cvitems} % Description(s) of tasks/responsibilities
        \item {Implemented an anagram finder which for a given input loads up all the anagrams from a separate Vocabulary file.
}
        \item {Code was made to operate faster and more efficiently by implementing
our own Hash Functions
for faster retrieval.}
	\item {
     Implemented B Trees in JAVA used for Data indexing.
    }
      \end{cvitems}
      %\begin{cvsubentries}
      %  \cvsubentry{}{KNOX(Solution for Enterprise Mobile Security) Penetration Testing}{Sep. 2013}{}
      %  \cvsubentry{}{Smart TV Penetration Testing}{Mar. 2011 - Oct. 2011}{}
      %\end{cvsubentries}
    }

%---------------------------------------------------------
\end{cventries}
